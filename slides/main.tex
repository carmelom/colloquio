\documentclass[10pt]{beamer}

\usepackage{style-custom}


\title{Superfluid properties of exciton polariton condensates in planar microcavities}
\subtitle{Colloquio IV anno}
\author[Carmelo Mordini]{Carmelo Mordini\\ \vspace{.7cm}{\tiny Supervisors\\ \vspace{-.2cm}I. Carusotto ~ R. Fazio}}

\institute[SNS] 
{
  Scuola Normale Superiore\\
  %University of Somewhere
  }
\date{Maggio 2014}


% % Delete this, if you do not want the table of contents to pop up at
% % the beginning of each subsection:
% \AtBeginSubsection[]
% {
%   \begin{frame}<beamer>{Riassunto}
%     \tableofcontents[currentsection,currentsubsection]
%   \end{frame}
% }

% Let's get started


\begin{document}

\begin{frame}
  \titlepage
\end{frame}

\begin{frame}{Outline}
  \tableofcontents[pausesections]
  % You might wish to add the option [pausesections]
\end{frame}

% Section and subsections will appear in the presentation overview
% and table of contents.
\section{Cavità di semiconduttori}

\subsection{Dinamica dei polaritoni}

 \begin{frame}{Cavità 2D}
 \transwipe[direction=270]
 \small

\begin{minipage}{\textwidth}
  \begin{columns}
  
  \begin{column}{.6\textwidth}
    \includegraphics[scale=.2]{files/QW_color.png}
  \end{column}

  \begin{column}{.4\textwidth}
    Le cavità bidimensionali di semiconduttore si rivelano uno strumento potente per studiare la fisica dei fluidi quantistici.
    \end{column}
  \end{columns}
  \end{minipage}
  
  Il confinamento spaziale e l'interazione con il mezzo fanno acquistare alla luce una dinamica analoga a quella di un fluido di particelle.
  
  \begin{minipage}{\textwidth}
  \begin{columns}
  
  \begin{column}{.5\textwidth}
    \begin{itemize}
     \item Polaritoni
     \item Superfluidità
     \item Esperimenti
    \end{itemize}

  \end{column}

  \begin{column}{.5\textwidth}
  \begin{figure}
    \includegraphics[scale=.2]{files/Panda.jpg}
  \end{figure}
    \end{column}
  \end{columns}
  \end{minipage}
  
  
 \end{frame}


\begin{frame}{Cavità 2D}%{Optional Subtitle}

    \begin{figure}
     \fbox{\includegraphics[scale=.7]{files/QW_field.png}}
    \end{figure}
    
    \begin{itemize}
     \item Massa del fotone dipendente dallo stato trasverso eccitato
    \end{itemize}
\vspace{-.5cm}        
      \begin{minipage}{\textwidth}
      \begin{columns}
	  \begin{column}{.5\textwidth}

	   \begin{align*}
	\omega_C(k) &= \frac{c}{n_0}\sqrt{{q_z}^2 + k^2} \\ 
		&\simeq \omega_C^0 + \frac{\hbar k^2}{2m_C}
	\end{align*}
	  \end{column}
	    
	  \begin{column}{.5\textwidth}
	  \footnotesize
	    $$\begin{cases}
	      q_z = \frac{\pi M}{\ell_z} \\
	      m_C = \frac{\hbar n_0 q_z}{c} = \frac{\hbar \omega_C^0}{c^2/n_0^2}
	    \end{cases}
	    $$
	   \end{column}
      \end{columns}

      \end{minipage}

  

    
\end{frame}

 \begin{frame}{Quantum Wells}
 Questa slide fa schifo
 \begin{figure}
  \includegraphics[scale=.5]{files/QW.jpg}
 \end{figure}
  $$\omega_X(k) = \omega_X^0 + \frac{\hbar k^2}{2m_X}$$
  \end{frame}

 

\begin{frame}{Dinamica}


  Gradi di libertà sul piano $xy$
  \begin{itemize}[<+->]
  
  \item { fotoni
   \begin{flalign*}
   \ham_{cav} = \intk \hbar \omega_C (k) \ \oa_C^\dagger \oa_C  &&
  \end{flalign*}
  }
  
    \item { eccitoni, con interaz. di dipolo (RWA)
 \begin{flalign*} 
    \ham_{exc} = \intk \hbar \omega_X (k) \ \oa_X^\dagger  \oa_X ~ + ~\hbar \Omega_R \left(\oa_C^\dagger \oa_X + \oa_X^\dagger \oa_C\right) &&
 \end{flalign*}
   }
  \end{itemize}
  
  \onslide<\thebeamerpauses>{
  \transdissolve<\thebeamerpauses>
 \begin{equation*}
  \boxed{
   \ham_{free} = \hbar \displaystyle \intk 
      \begin{pmatrix} \oa_C^\dagger & \oa_X^\dagger \end{pmatrix}\,
      \begin{pmatrix} \omega_C & \Omega_R \\ \Omega_R & \omega_X \end{pmatrix}\,\begin{pmatrix}\oa_C \\ \oa_X \end{pmatrix}
      }
  \end{equation*}
  }
\end{frame}

\begin{frame}{Polaritoni liberi}

\begin{equation*}
%\begin{align}
 \ham_{free} = \hbar \intk \omega_{LP} (k) \oa_{LP}^\dagger \oa_{LP} ~ + ~ \omega_{UP} (k) \oa_{UP}^\dagger \oa_{UP}
%\end{align}
\end{equation*}

\begin{minipage}{\textwidth}
\begin{columns}
\footnotesize{
  \begin{column}{.4\textwidth}
  
   $$\begin{cases}
	\oa_C = C\lp \ \oa\lp + C\up \ \oa\up \\
	\oa_X = X\lp \ \oa\lp + X\up \ \oa\up
     \end{cases}
   $$
   Coefficienti di Hopfield

   \begin{equation*}
    \begin{align}
      &\omega_{\text{\tiny{$(UP,LP)$}}} (k) = \frac{\omega_C (k) + \omega_X (k)}{2} \\
			     &\pm \left[\Omega_R^2 + \left(\frac{\omega_C (k) - \omega_X (k)}{2}\right)^2\right]^{1/2}
    \end{align}
   \end{equation*}
   Dispersione
  
  \end{column}
  }
  \hspace{.5cm}
  \begin{column}{.55\textwidth}
   \begin{figure}[h]
    \includegraphics[scale=.2]{files/polariton_dispersion.png}
   \end{figure}

  \end{column}
\end{columns}
\end{minipage}

\end{frame}



\subsection{Interazioni}

% You can reveal the parts of a slide one at a time
% with the \pause command:
\begin{frame}{Interazioni}
  %\transwipe[direction=270]
  Di carattere fotonico ed elettronico\\
  \footnotesize
  (le ultime sono dominanti)
  \normalsize
  \vskip.5cm
  \begin{itemize}
   \item {
   Nonlinearità ottiche
	  \begin{flalign*}
	   \chi^{(3)}{\vec E_{cav}}^4 \propto \chi^{(3)}(\oa_C^\dagger + \oa_C)^4 &&
	  \end{flalign*}
   }
   \item{
   Scattering coulombiano tra eccitoni
	    \begin{flalign*}
	     \text{\large{$\tilde V$}} (k,k',q) \ \oa_X^\dagger (k+q) \oa_X^\dagger (k'-q) \oa_X (k') \oa_X (k) &&
	    \end{flalign*}
   }
  \end{itemize}

  \pause
	    
\begin{columns}
 \begin{column}
  {.8\textwidth}
  \begin{equation*}
    \boxed{
      \ham_{int} = \intr \sum_{j=X,C} \frac{\hbar g_j}{2} \ \opsi_j^\dagger (r) \opsi_j^\dagger (r) \opsi_j (r) \opsi_j (r)
        }
   \end{equation*}
 \end{column}
 
 \begin{column}
  {.2\textwidth}
  \begin{itemize}
   \item $ka \ll 1$
   \item RWA
  \end{itemize}

 \end{column}


\end{columns}

\end{frame}

\subsection{Pumping \& losses}

\begin{frame}{Pompaggio laser}
  \begin{columns}
 \begin{column}
  {.45\textwidth}
    \begin{figure}[t]
    \flushleft
     \includegraphics[scale=.18]{files/incoherent.png}
    \end{figure}

 \end{column}
 
 \begin{column}
  {.55\textwidth}

     \textbf{Non coerente}\\
     Alto detunig; Rilassamento; Altro...\\
     Fase del modo LP non fissata\\
     \vskip15pt
     \textbf{Coerente}\\
     Fase del condensato fissata dal laser\\
     si può includere della dinamica
  
 \end{column}
\end{columns}

 \begin{equation*}
 \boxed{
    \begin{align*}
       \ham_{pump} &=\intr i\hbar \ \eta E_{inc} \ e^{ik_p r -i\omega_p t} \ \opsi_C^\dagger (r) +\hc \\
	&= \quad i\hbar \ \eta E_{inc} \ e^{-i\omega_p t} \ \oa_C^\dagger (k_p) +\hc
    \end{align*}
    }
 \end{equation*}
\end{frame}


\begin{frame}{Pompaggio laser}

\begin{columns}
\hskip10pt
\begin{column}{.4\textwidth}
      
\LaTeX~ è stupido\\
qui devo dire che ci restringiamo al solo LP

  \end{column}
  
  \begin{column}{.6\textwidth}
      \begin{figure}
       \includegraphics[scale=.3]{files/LP_pump.png}
      \end{figure}
  \end{column}   
 
\end{columns}

\end{frame}


\begin{frame}{Dissipazione}
MASTER EQUATION \alert{SUPER POWWA}!!!111!!!ONE11!
\begin{equation*}
i\hbar \partial_t\rho = [H,\rho] + i\hbar \lind \rho
\end{equation*}

dove l'operatore di Lindblad è solo quello radiativo:
\begin{equation*}
\lind \rho = \intk \frac{\gamma_{{\scriptscriptstyle rad}}}{2} \left ( 2\oa_C\rho \oa_C^\dagger - \oa_C^\dagger \oa_C \rho - \rho \oa_C^\dagger \oa_C \right )
\end{equation*}

\begin{figure}
       \includegraphics[scale=.3]{files/Panda.jpg}
       \caption{\footnotesize Qui ci rimetto il panda, che almeno lui mi fa compagnia}
      \end{figure}
 
\end{frame}


\section{Teoria di campo medio}

\begin{frame}{ddGPE}

\[ i\hbar \partial_t \trcurl{\rho \opsi\lp} \approx i\hbar \partial_t \psi\lp\]

%{\centering
Approx: \begin{flalign*} \begin{cases}
	     \left< \opsi^\dagger \opsi\, \opsi \right> \approx |\psi|^2 \psi \\
	     \omega\lp (k) \approx {\omega\lp}^0 + \displaystyle\frac{\hbar k^2}{2m\lp} \\
           \end{cases}
           &&
           \end{flalign*}


 

 \[ 
 \boxed{
 i\partial_t \, \psi\lp = \left[ \omega\lp (-i\nabla) + g\lp |\psi\lp|^2 - i \frac{\gamma\lp}{2} \right]\, \psi\lp + i F_p e^{ik_pr - \omega_p t}
 }
\]

\begin{flalign*}
 \begin{cases}
  m\lp = m_C/|C\lp|^2 \qquad &g\lp = |X\lp|^2 g_X + |C\lp|^2 g_C \\
  \gamma\lp = |C\lp|^2 \gamma_{{\scriptscriptstyle rad}} \qquad &F_p = {C\lp}^*\, \eta E_{inc}
 \end{cases}
&&
\end{flalign*}

 

 

\end{frame}

\subsection{Stato stazionario}

\begin{frame}{Roba}
%\def\number{1}

\ifnum\number=1
{
\documentclass{article}
\usepackage[utf8]{inputenc}

\title{panda}
\author{Carmelo Mordini}
\date{May 2014}

\usepackage[utf8]{inputenc} % for writing other that basic characters
\usepackage[italian]{babel}
\usepackage{amsmath}
\usepackage{graphicx}
\usepackage{caption}
\usepackage{subcaption}

\begin{document}
}
\fi

\begin{figure}[!ht]
        \centering
        \subfloat[Panda]{
                \includegraphics[trim=0.3cm 0cm 0cm 0cm, clip=true, 
                                width=.4\linewidth]
                                {files/pandabear/panda.jpg}
                       }
        \hfill
        \subfloat{
        \centering
        {\(\Rightarrow\)}
        }
        \hfill
        \subfloat[Polar bear]{
                \centering
                \includegraphics[trim=0cm 0cm 0cm 0cm, clip=true, 
                                width=.4\linewidth]
                                {files/pandabear/polarbear.jpg}
                       }
        \caption[The panda-polar bear relationship ]
                {It is not widely known that the panda becomes a polar bear 
                when dressing up in the winter camouflage suite}
        \label{fig: pentagram}
\end{figure}

\ifnum\number=1
{ \end{document}
}



\end{frame}


\begin{frame}{Blocks}
\begin{block}{Block Title}
You can also highlight sections of your presentation in a block, with it's own title
\end{block}
\begin{theorem}
There are separate environments for theorems, examples, definitions and proofs.
\end{theorem}
\begin{example}
Here is an example of an example block.
\end{example}
\end{frame}

\subsection{Spettro delle eccitazioni}
\begin{frame}{Altra roba}
\end{frame}


\section{Superfluidità}
\begin{frame}{Slide con foto fighe}
 
\end{frame}

\subsection{Resonant Raileigh Scattering}
\subsection{Criterio di Landau}
\subsection{Verifica sperimentale}

% Placing a * after \section means it will not show in the
% outline or table of contents.
\section*{Summary}

\begin{frame}{Summary}
  \begin{itemize}
  \item
    Colloquio \alert{superfigo}.
  \item
    \alert{Fazio in vacanza} senza di noi.
  \item
    Perhaps a \alert{third message}, but not more than that.
  \end{itemize}
  
  \begin{itemize}
  \item
    Outlook
    \begin{itemize}
    \item
      Something you haven't solved.
    \item
      Something else you haven't solved.
    \end{itemize}
  \end{itemize}
\end{frame}



% All of the following is optional and typically not needed. 
\appendix
\section<presentation>*{\appendixname}
\subsection<presentation>*{For Further Reading}

\begin{frame}[allowframebreaks]
  \frametitle<presentation>{For Further Reading}
    
  \begin{thebibliography}{10}
    
  \beamertemplatebookbibitems
  % Start with overview books.

  \bibitem{Author1990}
    A.~Author.
    \newblock {\em Handbook of Everything}.
    \newblock Some Press, 1990.

    
  \beamertemplatearticlebibitems
  % Followed by interesting articles. Keep the list short. 

  \bibitem{Someone2000}
    S.~Someone.
    \newblock On this and that.
    \newblock {\em Journal of This and That}, 2(1):50--100,
    2000.
  \end{thebibliography}
\end{frame}

\end{document}


